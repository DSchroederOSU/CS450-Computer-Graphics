\documentclass[10pt, draftclsnofoot, onecolumn]{IEEEtran}

% This might mess up formatting
\setlength{\parindent}{0pt}

\usepackage{graphicx}
\usepackage{amssymb}
\usepackage{amsmath}
\usepackage{amsthm}

\usepackage{alltt}
\usepackage{float}
\usepackage{color}
\usepackage{url}
\usepackage{hyperref}
%%\usepackage[hyphenbreaks]{breakurl}
\usepackage{listings}

\usepackage{balance}
\usepackage[TABBOTCAP, tight]{subfigure}
\usepackage{enumitem}
\usepackage{pstricks, pst-node}

\usepackage{geometry}
\geometry{textheight=8.5in, textwidth=6in}

\newcommand{\cred}[1]{{\color{red}#1}}
\newcommand{\cblue}[1]{{\color{blue}#1}}

\usepackage{hyperref}


\lstdefinestyle{customc}{
  belowcaptionskip=1\baselineskip,
  breaklines=true,
  frame=L,
  xleftmargin=\parindent,
  language=C,
  showstringspaces=false,
  basicstyle=\footnotesize\ttfamily,
  keywordstyle=\bfseries\color{green!40!black},
  commentstyle=\itshape\color{purple!40!black},
  identifierstyle=\color{blue},
  stringstyle=\color{orange},
}

\def\name{Daniel Schroeder}

%pull in the necessary preamble matter for pygments output
\input{pygments.tex}

%% The following metadata will show up in the PDF properties
\hypersetup{
  colorlinks = true,
  urlcolor = black,
  pdfauthor = {\name},
  pdfkeywords = {CS444 ``Operating Systems'' memory management},
  pdftitle = {CS 444 Writing Assignment 3},
  pdfsubject = {CS 444 Writing Assignment 3},
  pdfpagemode = UseNone
}

\begin{document}

\begin{titlepage}
        \textbf{Daniel Schroeder}\\
        \textbf{schrodan@oregonstate.edu}\\
        \textbf{Final Project Proposal}\\
        \vspace{1.5cm}
        
        For my final project, I would like to create a city-scape where the user can ``move'' around and traverse the streets. Similar to the video we saw flying over Chicago, I would like to create a scene with buildings of different shapes and sizes but with the eye positing at ground level. This program will involve shaders for coloring the buildings, geometric modeling for generating the buildings, and more versatile use of gluLookAt and eye position techniques to allow the user to ``move'' and ``look'' around.
        I think this project should be worth 200 points because it combines a lot of different aspects from previous programs and applies the different techniques to multiple objects (buildings) in order to create a large scene. I've included a crude image of what the city-scape might look like.

        \vspace{1.5cm}
        \begin{center} 
		\begin{figure}[H]
            \centering
            \includegraphics[width=8cm,height=8cm]{city.eps}
        \end{figure}

    \end{center}
\end{titlepage}
\end{document}
